
% Default to the notebook output style

    


% Inherit from the specified cell style.




    
\documentclass[11pt]{article}

    
    
    \usepackage[T1]{fontenc}
    % Nicer default font (+ math font) than Computer Modern for most use cases
    \usepackage{mathpazo}

    % Basic figure setup, for now with no caption control since it's done
    % automatically by Pandoc (which extracts ![](path) syntax from Markdown).
    \usepackage{graphicx}
    % We will generate all images so they have a width \maxwidth. This means
    % that they will get their normal width if they fit onto the page, but
    % are scaled down if they would overflow the margins.
    \makeatletter
    \def\maxwidth{\ifdim\Gin@nat@width>\linewidth\linewidth
    \else\Gin@nat@width\fi}
    \makeatother
    \let\Oldincludegraphics\includegraphics
    % Set max figure width to be 80% of text width, for now hardcoded.
    \renewcommand{\includegraphics}[1]{\Oldincludegraphics[width=.8\maxwidth]{#1}}
    % Ensure that by default, figures have no caption (until we provide a
    % proper Figure object with a Caption API and a way to capture that
    % in the conversion process - todo).
    \usepackage{caption}
    \DeclareCaptionLabelFormat{nolabel}{}
    \captionsetup{labelformat=nolabel}

    \usepackage{adjustbox} % Used to constrain images to a maximum size 
    \usepackage{xcolor} % Allow colors to be defined
    \usepackage{enumerate} % Needed for markdown enumerations to work
    \usepackage{geometry} % Used to adjust the document margins
    \usepackage{amsmath} % Equations
    \usepackage{amssymb} % Equations
    \usepackage{textcomp} % defines textquotesingle
    % Hack from http://tex.stackexchange.com/a/47451/13684:
    \AtBeginDocument{%
        \def\PYZsq{\textquotesingle}% Upright quotes in Pygmentized code
    }
    \usepackage{upquote} % Upright quotes for verbatim code
    \usepackage{eurosym} % defines \euro
    \usepackage[mathletters]{ucs} % Extended unicode (utf-8) support
    \usepackage[utf8x]{inputenc} % Allow utf-8 characters in the tex document
    \usepackage{fancyvrb} % verbatim replacement that allows latex
    \usepackage{grffile} % extends the file name processing of package graphics 
                         % to support a larger range 
    % The hyperref package gives us a pdf with properly built
    % internal navigation ('pdf bookmarks' for the table of contents,
    % internal cross-reference links, web links for URLs, etc.)
    \usepackage{hyperref}
    \usepackage{longtable} % longtable support required by pandoc >1.10
    \usepackage{booktabs}  % table support for pandoc > 1.12.2
    \usepackage[inline]{enumitem} % IRkernel/repr support (it uses the enumerate* environment)
    \usepackage[normalem]{ulem} % ulem is needed to support strikethroughs (\sout)
                                % normalem makes italics be italics, not underlines
    

    
    
    % Colors for the hyperref package
    \definecolor{urlcolor}{rgb}{0,.145,.698}
    \definecolor{linkcolor}{rgb}{.71,0.21,0.01}
    \definecolor{citecolor}{rgb}{.12,.54,.11}

    % ANSI colors
    \definecolor{ansi-black}{HTML}{3E424D}
    \definecolor{ansi-black-intense}{HTML}{282C36}
    \definecolor{ansi-red}{HTML}{E75C58}
    \definecolor{ansi-red-intense}{HTML}{B22B31}
    \definecolor{ansi-green}{HTML}{00A250}
    \definecolor{ansi-green-intense}{HTML}{007427}
    \definecolor{ansi-yellow}{HTML}{DDB62B}
    \definecolor{ansi-yellow-intense}{HTML}{B27D12}
    \definecolor{ansi-blue}{HTML}{208FFB}
    \definecolor{ansi-blue-intense}{HTML}{0065CA}
    \definecolor{ansi-magenta}{HTML}{D160C4}
    \definecolor{ansi-magenta-intense}{HTML}{A03196}
    \definecolor{ansi-cyan}{HTML}{60C6C8}
    \definecolor{ansi-cyan-intense}{HTML}{258F8F}
    \definecolor{ansi-white}{HTML}{C5C1B4}
    \definecolor{ansi-white-intense}{HTML}{A1A6B2}

    % commands and environments needed by pandoc snippets
    % extracted from the output of `pandoc -s`
    \providecommand{\tightlist}{%
      \setlength{\itemsep}{0pt}\setlength{\parskip}{0pt}}
    \DefineVerbatimEnvironment{Highlighting}{Verbatim}{commandchars=\\\{\}}
    % Add ',fontsize=\small' for more characters per line
    \newenvironment{Shaded}{}{}
    \newcommand{\KeywordTok}[1]{\textcolor[rgb]{0.00,0.44,0.13}{\textbf{{#1}}}}
    \newcommand{\DataTypeTok}[1]{\textcolor[rgb]{0.56,0.13,0.00}{{#1}}}
    \newcommand{\DecValTok}[1]{\textcolor[rgb]{0.25,0.63,0.44}{{#1}}}
    \newcommand{\BaseNTok}[1]{\textcolor[rgb]{0.25,0.63,0.44}{{#1}}}
    \newcommand{\FloatTok}[1]{\textcolor[rgb]{0.25,0.63,0.44}{{#1}}}
    \newcommand{\CharTok}[1]{\textcolor[rgb]{0.25,0.44,0.63}{{#1}}}
    \newcommand{\StringTok}[1]{\textcolor[rgb]{0.25,0.44,0.63}{{#1}}}
    \newcommand{\CommentTok}[1]{\textcolor[rgb]{0.38,0.63,0.69}{\textit{{#1}}}}
    \newcommand{\OtherTok}[1]{\textcolor[rgb]{0.00,0.44,0.13}{{#1}}}
    \newcommand{\AlertTok}[1]{\textcolor[rgb]{1.00,0.00,0.00}{\textbf{{#1}}}}
    \newcommand{\FunctionTok}[1]{\textcolor[rgb]{0.02,0.16,0.49}{{#1}}}
    \newcommand{\RegionMarkerTok}[1]{{#1}}
    \newcommand{\ErrorTok}[1]{\textcolor[rgb]{1.00,0.00,0.00}{\textbf{{#1}}}}
    \newcommand{\NormalTok}[1]{{#1}}
    
    % Additional commands for more recent versions of Pandoc
    \newcommand{\ConstantTok}[1]{\textcolor[rgb]{0.53,0.00,0.00}{{#1}}}
    \newcommand{\SpecialCharTok}[1]{\textcolor[rgb]{0.25,0.44,0.63}{{#1}}}
    \newcommand{\VerbatimStringTok}[1]{\textcolor[rgb]{0.25,0.44,0.63}{{#1}}}
    \newcommand{\SpecialStringTok}[1]{\textcolor[rgb]{0.73,0.40,0.53}{{#1}}}
    \newcommand{\ImportTok}[1]{{#1}}
    \newcommand{\DocumentationTok}[1]{\textcolor[rgb]{0.73,0.13,0.13}{\textit{{#1}}}}
    \newcommand{\AnnotationTok}[1]{\textcolor[rgb]{0.38,0.63,0.69}{\textbf{\textit{{#1}}}}}
    \newcommand{\CommentVarTok}[1]{\textcolor[rgb]{0.38,0.63,0.69}{\textbf{\textit{{#1}}}}}
    \newcommand{\VariableTok}[1]{\textcolor[rgb]{0.10,0.09,0.49}{{#1}}}
    \newcommand{\ControlFlowTok}[1]{\textcolor[rgb]{0.00,0.44,0.13}{\textbf{{#1}}}}
    \newcommand{\OperatorTok}[1]{\textcolor[rgb]{0.40,0.40,0.40}{{#1}}}
    \newcommand{\BuiltInTok}[1]{{#1}}
    \newcommand{\ExtensionTok}[1]{{#1}}
    \newcommand{\PreprocessorTok}[1]{\textcolor[rgb]{0.74,0.48,0.00}{{#1}}}
    \newcommand{\AttributeTok}[1]{\textcolor[rgb]{0.49,0.56,0.16}{{#1}}}
    \newcommand{\InformationTok}[1]{\textcolor[rgb]{0.38,0.63,0.69}{\textbf{\textit{{#1}}}}}
    \newcommand{\WarningTok}[1]{\textcolor[rgb]{0.38,0.63,0.69}{\textbf{\textit{{#1}}}}}
    
    
    % Define a nice break command that doesn't care if a line doesn't already
    % exist.
    \def\br{\hspace*{\fill} \\* }
    % Math Jax compatability definitions
    \def\gt{>}
    \def\lt{<}
    % Document parameters
    \title{Linear Regression}
    
    
    

    % Pygments definitions
    
\makeatletter
\def\PY@reset{\let\PY@it=\relax \let\PY@bf=\relax%
    \let\PY@ul=\relax \let\PY@tc=\relax%
    \let\PY@bc=\relax \let\PY@ff=\relax}
\def\PY@tok#1{\csname PY@tok@#1\endcsname}
\def\PY@toks#1+{\ifx\relax#1\empty\else%
    \PY@tok{#1}\expandafter\PY@toks\fi}
\def\PY@do#1{\PY@bc{\PY@tc{\PY@ul{%
    \PY@it{\PY@bf{\PY@ff{#1}}}}}}}
\def\PY#1#2{\PY@reset\PY@toks#1+\relax+\PY@do{#2}}

\expandafter\def\csname PY@tok@w\endcsname{\def\PY@tc##1{\textcolor[rgb]{0.73,0.73,0.73}{##1}}}
\expandafter\def\csname PY@tok@c\endcsname{\let\PY@it=\textit\def\PY@tc##1{\textcolor[rgb]{0.25,0.50,0.50}{##1}}}
\expandafter\def\csname PY@tok@cp\endcsname{\def\PY@tc##1{\textcolor[rgb]{0.74,0.48,0.00}{##1}}}
\expandafter\def\csname PY@tok@k\endcsname{\let\PY@bf=\textbf\def\PY@tc##1{\textcolor[rgb]{0.00,0.50,0.00}{##1}}}
\expandafter\def\csname PY@tok@kp\endcsname{\def\PY@tc##1{\textcolor[rgb]{0.00,0.50,0.00}{##1}}}
\expandafter\def\csname PY@tok@kt\endcsname{\def\PY@tc##1{\textcolor[rgb]{0.69,0.00,0.25}{##1}}}
\expandafter\def\csname PY@tok@o\endcsname{\def\PY@tc##1{\textcolor[rgb]{0.40,0.40,0.40}{##1}}}
\expandafter\def\csname PY@tok@ow\endcsname{\let\PY@bf=\textbf\def\PY@tc##1{\textcolor[rgb]{0.67,0.13,1.00}{##1}}}
\expandafter\def\csname PY@tok@nb\endcsname{\def\PY@tc##1{\textcolor[rgb]{0.00,0.50,0.00}{##1}}}
\expandafter\def\csname PY@tok@nf\endcsname{\def\PY@tc##1{\textcolor[rgb]{0.00,0.00,1.00}{##1}}}
\expandafter\def\csname PY@tok@nc\endcsname{\let\PY@bf=\textbf\def\PY@tc##1{\textcolor[rgb]{0.00,0.00,1.00}{##1}}}
\expandafter\def\csname PY@tok@nn\endcsname{\let\PY@bf=\textbf\def\PY@tc##1{\textcolor[rgb]{0.00,0.00,1.00}{##1}}}
\expandafter\def\csname PY@tok@ne\endcsname{\let\PY@bf=\textbf\def\PY@tc##1{\textcolor[rgb]{0.82,0.25,0.23}{##1}}}
\expandafter\def\csname PY@tok@nv\endcsname{\def\PY@tc##1{\textcolor[rgb]{0.10,0.09,0.49}{##1}}}
\expandafter\def\csname PY@tok@no\endcsname{\def\PY@tc##1{\textcolor[rgb]{0.53,0.00,0.00}{##1}}}
\expandafter\def\csname PY@tok@nl\endcsname{\def\PY@tc##1{\textcolor[rgb]{0.63,0.63,0.00}{##1}}}
\expandafter\def\csname PY@tok@ni\endcsname{\let\PY@bf=\textbf\def\PY@tc##1{\textcolor[rgb]{0.60,0.60,0.60}{##1}}}
\expandafter\def\csname PY@tok@na\endcsname{\def\PY@tc##1{\textcolor[rgb]{0.49,0.56,0.16}{##1}}}
\expandafter\def\csname PY@tok@nt\endcsname{\let\PY@bf=\textbf\def\PY@tc##1{\textcolor[rgb]{0.00,0.50,0.00}{##1}}}
\expandafter\def\csname PY@tok@nd\endcsname{\def\PY@tc##1{\textcolor[rgb]{0.67,0.13,1.00}{##1}}}
\expandafter\def\csname PY@tok@s\endcsname{\def\PY@tc##1{\textcolor[rgb]{0.73,0.13,0.13}{##1}}}
\expandafter\def\csname PY@tok@sd\endcsname{\let\PY@it=\textit\def\PY@tc##1{\textcolor[rgb]{0.73,0.13,0.13}{##1}}}
\expandafter\def\csname PY@tok@si\endcsname{\let\PY@bf=\textbf\def\PY@tc##1{\textcolor[rgb]{0.73,0.40,0.53}{##1}}}
\expandafter\def\csname PY@tok@se\endcsname{\let\PY@bf=\textbf\def\PY@tc##1{\textcolor[rgb]{0.73,0.40,0.13}{##1}}}
\expandafter\def\csname PY@tok@sr\endcsname{\def\PY@tc##1{\textcolor[rgb]{0.73,0.40,0.53}{##1}}}
\expandafter\def\csname PY@tok@ss\endcsname{\def\PY@tc##1{\textcolor[rgb]{0.10,0.09,0.49}{##1}}}
\expandafter\def\csname PY@tok@sx\endcsname{\def\PY@tc##1{\textcolor[rgb]{0.00,0.50,0.00}{##1}}}
\expandafter\def\csname PY@tok@m\endcsname{\def\PY@tc##1{\textcolor[rgb]{0.40,0.40,0.40}{##1}}}
\expandafter\def\csname PY@tok@gh\endcsname{\let\PY@bf=\textbf\def\PY@tc##1{\textcolor[rgb]{0.00,0.00,0.50}{##1}}}
\expandafter\def\csname PY@tok@gu\endcsname{\let\PY@bf=\textbf\def\PY@tc##1{\textcolor[rgb]{0.50,0.00,0.50}{##1}}}
\expandafter\def\csname PY@tok@gd\endcsname{\def\PY@tc##1{\textcolor[rgb]{0.63,0.00,0.00}{##1}}}
\expandafter\def\csname PY@tok@gi\endcsname{\def\PY@tc##1{\textcolor[rgb]{0.00,0.63,0.00}{##1}}}
\expandafter\def\csname PY@tok@gr\endcsname{\def\PY@tc##1{\textcolor[rgb]{1.00,0.00,0.00}{##1}}}
\expandafter\def\csname PY@tok@ge\endcsname{\let\PY@it=\textit}
\expandafter\def\csname PY@tok@gs\endcsname{\let\PY@bf=\textbf}
\expandafter\def\csname PY@tok@gp\endcsname{\let\PY@bf=\textbf\def\PY@tc##1{\textcolor[rgb]{0.00,0.00,0.50}{##1}}}
\expandafter\def\csname PY@tok@go\endcsname{\def\PY@tc##1{\textcolor[rgb]{0.53,0.53,0.53}{##1}}}
\expandafter\def\csname PY@tok@gt\endcsname{\def\PY@tc##1{\textcolor[rgb]{0.00,0.27,0.87}{##1}}}
\expandafter\def\csname PY@tok@err\endcsname{\def\PY@bc##1{\setlength{\fboxsep}{0pt}\fcolorbox[rgb]{1.00,0.00,0.00}{1,1,1}{\strut ##1}}}
\expandafter\def\csname PY@tok@kc\endcsname{\let\PY@bf=\textbf\def\PY@tc##1{\textcolor[rgb]{0.00,0.50,0.00}{##1}}}
\expandafter\def\csname PY@tok@kd\endcsname{\let\PY@bf=\textbf\def\PY@tc##1{\textcolor[rgb]{0.00,0.50,0.00}{##1}}}
\expandafter\def\csname PY@tok@kn\endcsname{\let\PY@bf=\textbf\def\PY@tc##1{\textcolor[rgb]{0.00,0.50,0.00}{##1}}}
\expandafter\def\csname PY@tok@kr\endcsname{\let\PY@bf=\textbf\def\PY@tc##1{\textcolor[rgb]{0.00,0.50,0.00}{##1}}}
\expandafter\def\csname PY@tok@bp\endcsname{\def\PY@tc##1{\textcolor[rgb]{0.00,0.50,0.00}{##1}}}
\expandafter\def\csname PY@tok@fm\endcsname{\def\PY@tc##1{\textcolor[rgb]{0.00,0.00,1.00}{##1}}}
\expandafter\def\csname PY@tok@vc\endcsname{\def\PY@tc##1{\textcolor[rgb]{0.10,0.09,0.49}{##1}}}
\expandafter\def\csname PY@tok@vg\endcsname{\def\PY@tc##1{\textcolor[rgb]{0.10,0.09,0.49}{##1}}}
\expandafter\def\csname PY@tok@vi\endcsname{\def\PY@tc##1{\textcolor[rgb]{0.10,0.09,0.49}{##1}}}
\expandafter\def\csname PY@tok@vm\endcsname{\def\PY@tc##1{\textcolor[rgb]{0.10,0.09,0.49}{##1}}}
\expandafter\def\csname PY@tok@sa\endcsname{\def\PY@tc##1{\textcolor[rgb]{0.73,0.13,0.13}{##1}}}
\expandafter\def\csname PY@tok@sb\endcsname{\def\PY@tc##1{\textcolor[rgb]{0.73,0.13,0.13}{##1}}}
\expandafter\def\csname PY@tok@sc\endcsname{\def\PY@tc##1{\textcolor[rgb]{0.73,0.13,0.13}{##1}}}
\expandafter\def\csname PY@tok@dl\endcsname{\def\PY@tc##1{\textcolor[rgb]{0.73,0.13,0.13}{##1}}}
\expandafter\def\csname PY@tok@s2\endcsname{\def\PY@tc##1{\textcolor[rgb]{0.73,0.13,0.13}{##1}}}
\expandafter\def\csname PY@tok@sh\endcsname{\def\PY@tc##1{\textcolor[rgb]{0.73,0.13,0.13}{##1}}}
\expandafter\def\csname PY@tok@s1\endcsname{\def\PY@tc##1{\textcolor[rgb]{0.73,0.13,0.13}{##1}}}
\expandafter\def\csname PY@tok@mb\endcsname{\def\PY@tc##1{\textcolor[rgb]{0.40,0.40,0.40}{##1}}}
\expandafter\def\csname PY@tok@mf\endcsname{\def\PY@tc##1{\textcolor[rgb]{0.40,0.40,0.40}{##1}}}
\expandafter\def\csname PY@tok@mh\endcsname{\def\PY@tc##1{\textcolor[rgb]{0.40,0.40,0.40}{##1}}}
\expandafter\def\csname PY@tok@mi\endcsname{\def\PY@tc##1{\textcolor[rgb]{0.40,0.40,0.40}{##1}}}
\expandafter\def\csname PY@tok@il\endcsname{\def\PY@tc##1{\textcolor[rgb]{0.40,0.40,0.40}{##1}}}
\expandafter\def\csname PY@tok@mo\endcsname{\def\PY@tc##1{\textcolor[rgb]{0.40,0.40,0.40}{##1}}}
\expandafter\def\csname PY@tok@ch\endcsname{\let\PY@it=\textit\def\PY@tc##1{\textcolor[rgb]{0.25,0.50,0.50}{##1}}}
\expandafter\def\csname PY@tok@cm\endcsname{\let\PY@it=\textit\def\PY@tc##1{\textcolor[rgb]{0.25,0.50,0.50}{##1}}}
\expandafter\def\csname PY@tok@cpf\endcsname{\let\PY@it=\textit\def\PY@tc##1{\textcolor[rgb]{0.25,0.50,0.50}{##1}}}
\expandafter\def\csname PY@tok@c1\endcsname{\let\PY@it=\textit\def\PY@tc##1{\textcolor[rgb]{0.25,0.50,0.50}{##1}}}
\expandafter\def\csname PY@tok@cs\endcsname{\let\PY@it=\textit\def\PY@tc##1{\textcolor[rgb]{0.25,0.50,0.50}{##1}}}

\def\PYZbs{\char`\\}
\def\PYZus{\char`\_}
\def\PYZob{\char`\{}
\def\PYZcb{\char`\}}
\def\PYZca{\char`\^}
\def\PYZam{\char`\&}
\def\PYZlt{\char`\<}
\def\PYZgt{\char`\>}
\def\PYZsh{\char`\#}
\def\PYZpc{\char`\%}
\def\PYZdl{\char`\$}
\def\PYZhy{\char`\-}
\def\PYZsq{\char`\'}
\def\PYZdq{\char`\"}
\def\PYZti{\char`\~}
% for compatibility with earlier versions
\def\PYZat{@}
\def\PYZlb{[}
\def\PYZrb{]}
\makeatother


    % Exact colors from NB
    \definecolor{incolor}{rgb}{0.0, 0.0, 0.5}
    \definecolor{outcolor}{rgb}{0.545, 0.0, 0.0}



    
    % Prevent overflowing lines due to hard-to-break entities
    \sloppy 
    % Setup hyperref package
    \hypersetup{
      breaklinks=true,  % so long urls are correctly broken across lines
      colorlinks=true,
      urlcolor=urlcolor,
      linkcolor=linkcolor,
      citecolor=citecolor,
      }
    % Slightly bigger margins than the latex defaults
    
    \geometry{verbose,tmargin=1in,bmargin=1in,lmargin=1in,rmargin=1in}
    
    

    \begin{document}
    
    
    \maketitle
    
    

    
    \hypertarget{linear-regression-beginner-theory}{%
\section{Linear Regression: Beginner
Theory}\label{linear-regression-beginner-theory}}

    This notebook is quite mathematical since we want to understand this
machine learning algorithm pretty well. We'll work with simple linear
regression as a starting point where we predict one variable from
another variable and then we'll look at multiple linear regression where
we can predict one output variable from multiple input variables.

    \hypertarget{the-main-concept-is-to-make-our-predictions-as-close-to-the-actual-values-as-possible.}{%
\subsection{The main concept is to make our predictions as close to the
actual values as
possible.}\label{the-main-concept-is-to-make-our-predictions-as-close-to-the-actual-values-as-possible.}}

So how can we make this happen?

We want to see how far away our predictions are from the actual values,
but this won't be a simple subtraction because sometimes the prediction
can be higher than the original and when we want to add up all the
differences, it can give us an answer that is close to zero.

In order to deal with this, we square the differences before adding up
the squared differences.

\[ E = \sum_{i=1}^{N}{(y_i - \hat{y_i})^2} \]

    In the equation above, \$ y\_i \$ represents our actual output values
and \$ \hat{y_i} \$ represents our prediction values.

A perfect linear regression model is where every single \$ \hat{y_i} \$
is equal to the \$ y\_i \$ . In reality, this will be pretty much
impossible, but what we can do is try and adjust the model's parameters
so it can try and minimize this error to be as close to zero as
possible.

So let's look at a simple linear regression model.

From high school, you are probably used to \[ y = mx + b \] . Linear
regression works the same exact way because you are working with two
parameters: the slope \$ m \$ and the bias term (intercept) \$ b \$ . We
want to turn this into what a formal machine learning regression model
looks like:

\[ \hat{y_i} = \beta_1 x_i + \beta_0   \]

In the above equation, \$ \beta\_1 \$ is the same as \$ m \$ and \$
\beta\_0 \$ is the same as the intercept \$ b \$ .

In some classes, you probably heard of the term, the line of best fit .
The line of best fit is essentially the regression equation which
minimizes the error as much as possible. You might be wondering, how
does one calculate this line of best fit? This requires some
multivariable calculus for predicting the output from one variable and
linear algebra for predicting the output from multiple input variables.

In this notebook, I won't go through the entire derivation, but I'll
show you the important steps to get to the line of the best fit.

\hypertarget{step-1}{%
\subsection{Step 1:}\label{step-1}}

Substitute the regression model for \$ \hat{y_i} \$ in the error (cost)
function.

\[ E = \sum_{i=1}^{N}{(y_i - \hat{y_i})^2} \space \space  \space \space \space \space \rightarrow \space \space \space \space \space \space E = \sum_{i=1}^{N}{(y_i - (\beta_1 x_i + \beta_0))^2} \]

\hypertarget{step-2}{%
\subsection{Step 2:}\label{step-2}}

Take partial derivative of the error function with respect to \$
\beta\_1 \$ and \$ \beta\_0 \$

\[ \frac{\partial{E}}{\partial{\beta_1}} \space \space  \space \space \space \space = \space \space  \space \space \space \space \sum_{i=1}^{N}{2(y_i - (\beta_1 x_i + \beta_0))(-x_i)} \space \space  \space \space \space \space \rightarrow \space \space \space \space \space \space \beta_1 \sum_{i=1}^{N}{{x_i}^2} + \beta_0 \sum_{i=1}^{N}{x_i} = \sum_{i=1}^{N}{x_i y_i} \]

\[ \frac{\partial{E}}{\partial{\beta_0}} \space \space  \space \space \space \space = \space \space  \space \space \space \space \sum_{i=1}^{N}{2(y_i - (\beta_1 x_i + \beta_0))(-1)} \space \space  \space \space \space \space \rightarrow \space \space \space \space \space \space \beta_1 \sum_{i=1}^{N}{x_i} + \beta_0 N = \sum_{i=1}^{N}{y_i} \]

\hypertarget{step-3}{%
\subsection{Step 3:}\label{step-3}}

Replace the summations with letters.

\[ C = \sum_{i=1}^{N}{{x_i}^2} \] \[ D = \sum_{i=1}^{N}{x_i} \]
\[ E = \sum_{i=1}^{N}{x_i y_i} \] \[ F = \sum_{i=1}^{N}{y_i} \]

\[ \rightarrow \]

\[ \beta_1 C + \beta_0 D = E \] \[ \space \]
\[ \beta_1 D + \beta_0 N = F \]

\hypertarget{step-4}{%
\subsection{Step 4:}\label{step-4}}

Solve system of equations.

For \$ \beta\_0 \$ :

\[ D * {\beta_1 C + \beta_0 D = E} \]

\[ C * {\beta_1 D + \beta_0 N = F} \]

\[ \beta_1 CD + \beta_0 D^2 = ED \] \[ \text{-}  \]\\
\[ \beta_0 CD + \beta_1 CN = CF \]

=

\[ \beta_0 ({D^2} - CN) = {ED - CF} \] \[ \space \]
\[ \beta_0 = \frac{ED - CF}{D^2 - CN} \space \space \space \rightarrow \space \space \space \frac{\sum_{i=1}^{N}{x_i y_i}\sum_{i=1}^{N}{x_i} - \sum_{i=1}^{N}{{x_i}^2} \sum_{i=1}^{N}{y_i}}{N\sum_{i=1}^{N}{{x_i}^2} -  {(\sum_{i=1}^{N}{x_i})}^2} \]

For \$ \beta\_1 \$ :

\[ \beta_1 C + \beta_0 D = E \] \[ \space \]
\[ \beta_1 D + \beta_0 N = F \]

\[ N * {\beta_0 C + \beta_1 D = E} \] \[ \space \]
\[ D * {\beta_0 D + \beta_1 N = F} \]

\[ \beta_1 CN + \beta_0 DN = EN \] \[ \space \] \[ \text{-} \]
\[ \space \] \[ \beta_1 D^2 + \beta_0 DN = FD \] \[ \space \] =
\[ \space \] \[ \beta_1 ({CN - D^2}) = {EN - FD} \] \[ \space \]
\[ \beta_1 = \frac{EN - FD}{CN - D^2} = \frac{FD - EN}{D^2 - CN} \space \space \space \rightarrow \space \space \space \frac{N\sum_{i=1}^{N}{x_i y_i} - \sum_{i=1}^{N}{x_i} \sum_{i=1}^{N}{y_i}}{N\sum_{i=1}^{N}{{x_i}^2} -  {(\sum_{i=1}^{N}{x_i})}^2} \]

\hypertarget{step-5}{%
\subsection{Step 5:}\label{step-5}}

Finding Relation to Sample Means:

Multiplying by 1 / N\^{}2 :

\[ \beta_1 * \frac{\frac{1}{N^2}}{\frac{1}{N^2}} = \frac{N\sum_{i=1}^{N}{x_i y_i} - \sum_{i=1}^{N}{x_i} \sum_{i=1}^{N}{y_i}}{N\sum_{i=1}^{N}{{x_i}^2} -  {(\sum_{i=1}^{N}{x_i})}^2} * \frac{\frac{1}{N^2}}{\frac{1}{N^2}} \]

\[ \beta_0 *\frac{\frac{1}{N^2}}{\frac{1}{N^2}} = \frac{\sum_{i=1}^{N}{x_i y_i}\sum_{i=1}^{N}{x_i} - \sum_{i=1}^{N}{{x_i}^2} \sum_{i=1}^{N}{y_i}}{N\sum_{i=1}^{N}{{x_i}^2} -  {(\sum_{i=1}^{N}{x_i})}^2} * \frac{\frac{1}{N^2}}{\frac{1}{N^2}} \]

\[ \beta_1 = \frac{\bar{xy} - \bar{x}\bar{y}}{{\bar{x^2}} - \bar{x}^2} \]

\[ \beta_0 = \frac{\bar{xy}\bar{x} - \bar{x^2}\bar{y}}{\bar{x^2} - \bar{x}^2} \]

    If we plug in these values of \$ \beta\_1 \$ and \$ \beta\_0 \$ to our
regression model, we will receive the line of the best fit.

    \hypertarget{linear-regression-example-in-numpy}{%
\section{Linear Regression Example in
Numpy}\label{linear-regression-example-in-numpy}}

    \begin{Verbatim}[commandchars=\\\{\}]
{\color{incolor}In [{\color{incolor}9}]:} \PY{k+kn}{import} \PY{n+nn}{numpy} \PY{k}{as} \PY{n+nn}{np}
        \PY{k+kn}{import} \PY{n+nn}{pandas} \PY{k}{as} \PY{n+nn}{pd} 
        \PY{o}{\PYZpc{}}\PY{k}{matplotlib} inline
        \PY{k+kn}{import} \PY{n+nn}{matplotlib}\PY{n+nn}{.}\PY{n+nn}{pyplot} \PY{k}{as} \PY{n+nn}{plt} 
        
        \PY{n}{data} \PY{o}{=} \PY{n}{pd}\PY{o}{.}\PY{n}{read\PYZus{}csv}\PY{p}{(}\PY{l+s+s2}{\PYZdq{}}\PY{l+s+s2}{../data/regression.csv}\PY{l+s+s2}{\PYZdq{}}\PY{p}{)}
        \PY{n}{data}\PY{o}{.}\PY{n}{head}\PY{p}{(}\PY{p}{)} 
        
        \PY{n}{X} \PY{o}{=} \PY{n}{data}\PY{p}{[}\PY{l+s+s1}{\PYZsq{}}\PY{l+s+s1}{X}\PY{l+s+s1}{\PYZsq{}}\PY{p}{]}
        \PY{n}{Y} \PY{o}{=} \PY{n}{data}\PY{p}{[}\PY{l+s+s1}{\PYZsq{}}\PY{l+s+s1}{Y}\PY{l+s+s1}{\PYZsq{}}\PY{p}{]} 
        
        \PY{n}{plt}\PY{o}{.}\PY{n}{figure}\PY{p}{(}\PY{n}{figsize}\PY{o}{=}\PY{p}{(}\PY{l+m+mi}{14}\PY{p}{,}\PY{l+m+mi}{8}\PY{p}{)}\PY{p}{)}
        \PY{n}{plt}\PY{o}{.}\PY{n}{title}\PY{p}{(}\PY{l+s+s2}{\PYZdq{}}\PY{l+s+s2}{Scatterplot of X and Y}\PY{l+s+s2}{\PYZdq{}}\PY{p}{,} \PY{n}{size}\PY{o}{=}\PY{l+s+s2}{\PYZdq{}}\PY{l+s+s2}{20}\PY{l+s+s2}{\PYZdq{}}\PY{p}{)}
        \PY{n}{plt}\PY{o}{.}\PY{n}{xlabel}\PY{p}{(}\PY{l+s+s2}{\PYZdq{}}\PY{l+s+s2}{Input}\PY{l+s+s2}{\PYZdq{}}\PY{p}{,} \PY{n}{size}\PY{o}{=}\PY{l+s+s2}{\PYZdq{}}\PY{l+s+s2}{15}\PY{l+s+s2}{\PYZdq{}}\PY{p}{)}
        \PY{n}{plt}\PY{o}{.}\PY{n}{ylabel}\PY{p}{(}\PY{l+s+s2}{\PYZdq{}}\PY{l+s+s2}{Output}\PY{l+s+s2}{\PYZdq{}}\PY{p}{,} \PY{n}{size}\PY{o}{=}\PY{l+s+s2}{\PYZdq{}}\PY{l+s+s2}{15}\PY{l+s+s2}{\PYZdq{}}\PY{p}{)}
        \PY{n}{plt}\PY{o}{.}\PY{n}{scatter}\PY{p}{(}\PY{n}{X}\PY{p}{,}\PY{n}{Y}\PY{p}{)}
\end{Verbatim}


\begin{Verbatim}[commandchars=\\\{\}]
{\color{outcolor}Out[{\color{outcolor}9}]:} <matplotlib.collections.PathCollection at 0x110dbd588>
\end{Verbatim}
            
    \begin{center}
    \adjustimage{max size={0.9\linewidth}{0.9\paperheight}}{output_6_1.png}
    \end{center}
    { \hspace*{\fill} \\}
    
    \begin{Verbatim}[commandchars=\\\{\}]
{\color{incolor}In [{\color{incolor}8}]:} \PY{c+c1}{\PYZsh{} Let\PYZsq{}s create the regression model }
        
        \PY{c+c1}{\PYZsh{} For our line of best fit, we know that the denominators for beta 1 and beta 0 are the same so let\PYZsq{}s set up a variable }
        \PY{n}{X} \PY{o}{=} \PY{n}{np}\PY{o}{.}\PY{n}{array}\PY{p}{(}\PY{n}{X}\PY{p}{)}
        \PY{n}{Y} \PY{o}{=} \PY{n}{np}\PY{o}{.}\PY{n}{array}\PY{p}{(}\PY{n}{Y}\PY{p}{)} 
        \PY{n}{denominator} \PY{o}{=} \PY{n}{X}\PY{o}{.}\PY{n}{dot}\PY{p}{(}\PY{n}{X}\PY{p}{)} \PY{o}{\PYZhy{}} \PY{n}{X}\PY{o}{.}\PY{n}{mean}\PY{p}{(}\PY{p}{)} \PY{o}{*} \PY{n}{X}\PY{o}{.}\PY{n}{sum}\PY{p}{(}\PY{p}{)} 
        \PY{n}{beta\PYZus{}1} \PY{o}{=} \PY{p}{(}\PY{n}{X}\PY{o}{.}\PY{n}{dot}\PY{p}{(}\PY{n}{Y}\PY{p}{)} \PY{o}{\PYZhy{}} \PY{n}{Y}\PY{o}{.}\PY{n}{mean}\PY{p}{(}\PY{p}{)}\PY{o}{*}\PY{n}{X}\PY{o}{.}\PY{n}{sum}\PY{p}{(}\PY{p}{)}\PY{p}{)} \PY{o}{/} \PY{n}{denominator} 
        \PY{n}{beta\PYZus{}0} \PY{o}{=} \PY{p}{(}\PY{n}{Y}\PY{o}{.}\PY{n}{mean}\PY{p}{(}\PY{p}{)} \PY{o}{*} \PY{n}{X}\PY{o}{.}\PY{n}{dot}\PY{p}{(}\PY{n}{X}\PY{p}{)} \PY{o}{\PYZhy{}} \PY{n}{X}\PY{o}{.}\PY{n}{mean}\PY{p}{(}\PY{p}{)} \PY{o}{*} \PY{n}{X}\PY{o}{.}\PY{n}{dot}\PY{p}{(}\PY{n}{Y}\PY{p}{)}\PY{p}{)} \PY{o}{/} \PY{n}{denominator} 
        
        \PY{c+c1}{\PYZsh{} our regression model }
        
        \PY{n}{y\PYZus{}hat} \PY{o}{=} \PY{n}{beta\PYZus{}1} \PY{o}{*} \PY{n}{X} \PY{o}{+} \PY{n}{beta\PYZus{}0} 
        
        \PY{c+c1}{\PYZsh{} Plotting the line of best fit }
        
        \PY{n}{plt}\PY{o}{.}\PY{n}{figure}\PY{p}{(}\PY{n}{figsize}\PY{o}{=}\PY{p}{(}\PY{l+m+mi}{14}\PY{p}{,}\PY{l+m+mi}{8}\PY{p}{)}\PY{p}{)}
        \PY{n}{plt}\PY{o}{.}\PY{n}{title}\PY{p}{(}\PY{l+s+s2}{\PYZdq{}}\PY{l+s+s2}{Scatterplot with Line of Best Fit}\PY{l+s+s2}{\PYZdq{}}\PY{p}{,} \PY{n}{size}\PY{o}{=}\PY{l+s+s2}{\PYZdq{}}\PY{l+s+s2}{20}\PY{l+s+s2}{\PYZdq{}}\PY{p}{)}
        \PY{n}{plt}\PY{o}{.}\PY{n}{xlabel}\PY{p}{(}\PY{l+s+s2}{\PYZdq{}}\PY{l+s+s2}{Input}\PY{l+s+s2}{\PYZdq{}}\PY{p}{,} \PY{n}{size}\PY{o}{=}\PY{l+s+s2}{\PYZdq{}}\PY{l+s+s2}{15}\PY{l+s+s2}{\PYZdq{}}\PY{p}{)}
        \PY{n}{plt}\PY{o}{.}\PY{n}{ylabel}\PY{p}{(}\PY{l+s+s2}{\PYZdq{}}\PY{l+s+s2}{Output}\PY{l+s+s2}{\PYZdq{}}\PY{p}{,} \PY{n}{size}\PY{o}{=}\PY{l+s+s2}{\PYZdq{}}\PY{l+s+s2}{15}\PY{l+s+s2}{\PYZdq{}}\PY{p}{)}
        \PY{n}{plt}\PY{o}{.}\PY{n}{scatter}\PY{p}{(}\PY{n}{X}\PY{p}{,}\PY{n}{Y}\PY{p}{)}
        \PY{n}{plt}\PY{o}{.}\PY{n}{plot}\PY{p}{(}\PY{n}{X}\PY{p}{,} \PY{n}{y\PYZus{}hat}\PY{p}{,} \PY{l+s+s1}{\PYZsq{}}\PY{l+s+s1}{g}\PY{l+s+s1}{\PYZsq{}}\PY{p}{)} 
\end{Verbatim}


\begin{Verbatim}[commandchars=\\\{\}]
{\color{outcolor}Out[{\color{outcolor}8}]:} [<matplotlib.lines.Line2D at 0x10d7311d0>]
\end{Verbatim}
            
    \begin{center}
    \adjustimage{max size={0.9\linewidth}{0.9\paperheight}}{output_7_1.png}
    \end{center}
    { \hspace*{\fill} \\}
    
    \hypertarget{assessing-our-model}{%
\subsection{Assessing Our Model}\label{assessing-our-model}}

For linear regression models, we assess the performance by looking at
essentially the accuracy metric which in our case is going to be known
as the \$ r\^{}2 \$ value.

The \$ r\^{}2 \$ value is calculated as follows:

\[ r^2 = \frac{1 - (y_i - \hat{y_i})^2}{(y_i - \bar{y_i})^2} \]

This value can be anywhere between 0 and 1. As a percentage, the closer
we are to 100\% shows a very accurate model.

    \begin{Verbatim}[commandchars=\\\{\}]
{\color{incolor}In [{\color{incolor}14}]:} \PY{c+c1}{\PYZsh{} Getting our r\PYZhy{}square value }
         
         \PY{n}{difference\PYZus{}one} \PY{o}{=} \PY{n}{Y} \PY{o}{\PYZhy{}} \PY{n}{y\PYZus{}hat} 
         \PY{n}{difference\PYZus{}two} \PY{o}{=} \PY{n}{Y} \PY{o}{\PYZhy{}} \PY{n}{Y}\PY{o}{.}\PY{n}{mean}\PY{p}{(}\PY{p}{)} 
         
         \PY{n}{r\PYZus{}square} \PY{o}{=} \PY{l+m+mi}{1} \PY{o}{\PYZhy{}} \PY{n}{difference\PYZus{}one}\PY{o}{.}\PY{n}{dot}\PY{p}{(}\PY{n}{difference\PYZus{}one}\PY{p}{)} \PY{o}{/} \PY{n}{difference\PYZus{}two}\PY{o}{.}\PY{n}{dot}\PY{p}{(}\PY{n}{difference\PYZus{}two}\PY{p}{)} 
         
         
         
         \PY{k+kn}{from} \PY{n+nn}{IPython}\PY{n+nn}{.}\PY{n+nn}{display} \PY{k}{import} \PY{n}{Markdown}\PY{p}{,} \PY{n}{display}
         
         \PY{n}{display}\PY{p}{(}\PY{n}{Markdown}\PY{p}{(}\PY{n}{f}\PY{l+s+s2}{\PYZdq{}}\PY{l+s+s2}{Our \PYZdl{} r\PYZca{}2 \PYZdl{} value is: }\PY{l+s+s2}{\PYZob{}}\PY{l+s+s2}{round(r\PYZus{}square*100,3)\PYZcb{}}\PY{l+s+s2}{\PYZpc{}}\PY{l+s+s2}{\PYZdq{}}\PY{p}{)}\PY{p}{)}
\end{Verbatim}


    Our \$ r\^{}2 \$ value is: 99.118\%

    

    % Add a bibliography block to the postdoc
    
    
    
    \end{document}
